\documentclass[titlepage]{article}
\usepackage{amsmath, amssymb}
\usepackage{minted}
\newmintinline[cinline]{c}{}
\newcommand{\N}{\mathbb{N}}
\newcommand{\R}{\mathbb{R}}
\newcommand{\I}{\mathbb{I}}
\renewcommand{\S}{\texttt{S}}
\newcommand{\xuniform}{\ensuremath{X_{\texttt{uniform}}}}
\newcommand{\zo}{\{0, 1\}}
\newcommand{\uniformbool}{\texttt{uniformbool}}
\newcommand{\p}{\mathbf{p}}
\newcommand{\q}{\mathbf{q}}
\title{A detailed reference of MCMC algorithms}
\author{Siddharth Bhat (20161105) ~ \texttt{siddu.druid@gmail.com}}
\date{\today}
\begin{document}
\maketitle
\section{Why do we need MCMC? A practitioner's perspective}
Consider that we are plonked down in the \texttt{C } programming language, and
our only method to generate random numbers is to call \cinline{int rand(void)}.
However, \emph{the type is a lie}, since we are able to modify global mutable
state. So, really, to have a mathematical discussion about the whole state
of affairs, we will write the type of \cinline{rand} as
$\cinline{rand}: S \rightarrow S \times \cinline{int}$ --- that is,
it receives the entire state of the program, and then returns the \cinline{int},
along with the new state of the program.
$\uniformbool: \S \rightarrow \S \times \zo$.

When we say that \texttt{rand} generates random numbers, we need to be a
little more specific: what does it mean to generate random numbers? we need to
describe the \emph{distribution} according to which we are receiving the random
numbers from the random number generator $rand$.  What does that mean?
Well, it means that as we generate more numbers, the \emph{empirical distribution}
of the list of numbers we get from the successive calls to $\uniformbool$ tends
to some \emph{true distribution}. We will call this \emph{true distribution}
succincty as \textbf{the} distribution of the random number generator. Formally,
let us define $F(t) \equiv \int_0^t P(x) dx$ to be the cumulative distribution 
of $P$.

In the case of 
$\uniformbool$, we are receiving random numbers according to the distribution:
$$
P_{\uniformbool}: \zo \rightarrow [0, 1]; \qquad P_{\uniformbool}(x) = 1/2
$$
That is, both $0$ and $1$ are \emph{equally likely}. However, this is
extremely boring. What we are \emph{usually} interested in is to sample $\{0, 1\}$
in some \emph{biased} fashion:
$$
P_{\uniformbool}^{bias}: \zo \rightarrow [0, 1]; 
\qquad P_{\uniformbool}^{bias}(0) = bias;
\qquad P_{\uniformbool}^{bias}(1) = 1 - bias
$$

And far more generally, we want to sample from \emph{arbitrary domains} with
\emph{arbitrary distributions}:

$$
\texttt{sampler}_X^P: S \rightarrow S \times X; 
$$

This function is boring. What we \emph{really} want to sample from are
more interesting distributions. For example:

\begin{itemize}
    \item The normal distribution $P(x: \R) = e^{-x^2}$.
    \item The poisson distribution $P(x: \N) = e^{-\lambda} \lambda^n/n!$.
    \item A custom distribution $P(x: \R) = |sin(x)|$.
\end{itemize}

\subsection{Fundamental Problem of MCMC sampling}
Given a weak, simple sampler of the form $\cinline{rand} : S \rightarrow S \times \cinline{int}$,
build a sampler $\texttt{sampler}(P, X): T \rightarrow T \times X$ which returns
value distributed according to some distribution of choice $P: X \rightarrow [0, 1]$.

\subsection{Sampling use case 1. Simulation And Sampling}
\subsection{Sampling use case 2. Gradient Free Optimisation}
\subsection{Sampling use case 3. Numerical Integration}

\section{Where it all begins: The Metropolis Hastings sampler}

\section{Gibbs sampling}
We wish to sample from a joint distribution $P(X_1, X_2, X_3)$. However, it might
be far cheaper to sample $P(X_1 | X_2, X_3)$, $P(X_2 | X_1, X_3)$, and
$P(X_3 | X_1, X_2)$. If it is indeed cheaper, then we can use a Gibbs sampler
to draw from the actual distribution $P(X_1, X_2, X_3)$ by combining samples
from the \emph{conditional} distribution cleverly. The code is:

\begin{minted}{py}
# sampler_x: y, z -> new x
# sampler_y: x, z -> new y
# sampler_z: x, y -> new z
# N: number of iterations.
# returns: new (x, y, z)
def sampler_xyz(sampler_x, sampler_y, sampler_z, x0, y0, z0, N):
    (x, y, z) = (x0, y0, z0)
    for i in range(N):
        x = sampler_x(y, z)
        y = sampler_y(x, z) # NOTE: use *new* x
        z = sampler_z(x, y) # NOTE: use *new* x, y
    return (x, y, z)
\end{minted}

\section{Hamiltonian Monte Carlo}
If our target probability density $P: X \rightarrow [0, 1]$ is \emph{differentiable},
then we can use the derivative of $P(x)$ to provide better proposals. The idea
is as follows:

\begin{itemize}
    \item Interpret the probability landscape as actual terrain, with
        points of high probability being "valleys" and points of low probability
        being "peaks" (ie, invert the probability density with a transform
        such as $terrian(x) = e^{-P(x)}$
    \item For a proposal at a position $x_0 \in X$, keep a ball at $x_0$,
        \emph{randomly choose its velocity}, simulate the ball according to 
        classical mechanics (Newton's laws of motion) for a fixed duration $D \in \R$
        and propose the final position as the final position of the ball. This
        is reversible and detail balanced because 
        \emph{classical mechanics is reversible and detail balanced}.
\end{itemize}
We need the derivative of $P(x)$ to be able to simulate the ball according
to Newton's laws. If we can do this, though, we are able to cover large
amounts of terrain.


\subsection{Newton's laws of motion: Hamilton's equations}

\begin{align*}
    &\frac{\partial \q}{\partial t} = \frac{\partial H}{\partial \p} \\
    &\frac{\partial \p}{\partial t} = - \frac{\partial H}{\partial \q}
\end{align*}

\subsection{Detail balance}



\subsection{Hamiltonian for our simulation}
If we have a a target probability distribution $Prob(\q): \R^n \rightarrow \R$
we choose the Hamiltonian to be $H(\p, \q) \equiv Prob(\q) + \p^T \p/2$.


\subsection{Simulation: Naive}
\subsection{Simulation: Need for symplectic integrators}
\subsection{Simulation: Final}


We no longer need to choose how many steps to walk with a no-U-turn-sampler.
It prevents us from re-walking energy orbits, by detecting when we have completed
traversing an orbit and are going to take a "U-turn". The details are quite
complex, so we may not cover this here.

\section{Discontinuous Hamiltonian monte carlo}

What if our distribution $P$ is \emph{discontinuouse}? How do we perform MCMC
in that case?

\section{Low discrepancy sequences}
% https://cseweb.ucsd.edu/~dstefan/pubs/dalal:2008:low.pdf
% https://openturns.github.io/openturns/master/theory/reliability_sensitivity/low_discrepancy_sequence.html
Low discrepancy sequences are sequences of numbers that more evenly
distributed than pseudorandom numbers in high dimensional space. Hence,
simulations which use Low discrepancy sequences generally approximate
integrals faster than psuedo-randomly generated points.

Formally, let us consider the $S$ dimensional half-open cube $\I^S \equiv [0, 1)^S$.
assume we have a set of points $P \subseteq \I^S$, and a sub-interval
$B \subseteq I^S$, where a sub-interval is a subset of the form 
$B \equiv \prod_{i=1}^S \{ x \in \I : a_i \leq x \leq b_i \}$.

Given a universe set $X$ and two subsets $Large, Small \subseteq X$, we define
the amount of containment of $Small$ in $Large$ to be $C(Small, Large) \equiv
\frac{|Large \cap Small|}{|Large|}$. Intuitively, this measures the fraction
of $Small$ that is in $Large$. We now define the discrepancy of the set
of points $P$ relative to the sub-interval $B$ as:

\begin{align*}
&D(B, P) \equiv \left| C(P, B) - C(B, \I^S) \right| \\
&= \left| \frac{|B \cap P|}{|P|} - \frac{|B \cap \I^S|}{|\I^S|} \right|  \\
&= \left| \frac{|B \cap P|}{|P|} - \frac{Volume(B)}{1} \right| \\
&= \left| \frac{|B \cap P|}{|P|} - Volume(B) \right| \\
\end{align*}

So, the discrepancy is measuring if $P$ fits within $B$ the way $B$ fits within
the full space.

Now, the \textbf{worst-case-discrepancy} is defined as the maximum discrepancy
over all sub-intervals:

$$
D^\star(P) \equiv \max_{B \in \mathcal{J}} D(B, P)
$$

where $\mathcal{J} \subseteq 2^{\I^S}$ is the set of all sub-intervals:

$$
\mathcal{J} \equiv \{ \{ x \in \I^S : l[i] \leq x[i] \leq r[i]~\forall i \} : \vec l, \vec r \in \I^S \}
$$

The goal is a \emph{low discrepancy sequence} is to minimise the worst-case-discrepancy.

\subsection{Error bounds for Numerical integration}

\subsection{Sobol sequences}

Sobol sequences are an example of low-discrepancy sequences


\section{Future Work!}
I've left the topic names up because they have interesting ideas, which I have
sketched out in brief. Writing them down formally will take far too long; Hence,
they've been left here [for the purposes of the project]. I will probably update
this document even after the project, once I have grokked this material better.
\section{Slice Sampling}
\section{No-U-Turn sampling}

\end{document}

